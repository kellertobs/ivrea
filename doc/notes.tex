
\documentclass[12pt,a4paper]{article}
\usepackage[margin=2cm]{geometry}
\usepackage[margin=5pt,font=small,labelfont=bf]{caption}
\usepackage{graphicx,amsmath,amssymb,natbib,setspace,lineno,xspace,color}
\definecolor{darkblue}{rgb}{0.0,0.0,0.6}
\usepackage[colorlinks=true,breaklinks=true,linkcolor=darkblue,urlcolor=darkblue,
            anchorcolor=darkblue,citecolor=darkblue]{hyperref}
%% DIFFERENTIAL EQUATIONS
\newcommand{\diff}[2]{\frac{d{#1}}{d{#2}}}
\newcommand{\pdiff}[2]{\frac{\partial{#1}}{\partial{#2}}}
\newcommand{\pdifftwo}[2]{\frac{\partial^2{#1}}{\partial{#2}^2}}
\newcommand{\pdiffmix}[3]{\frac{\partial^2{#1}}{\partial{#2}\partial{#3}}}
\newcommand{\pdiffcon}[3]{\left.\frac{\partial{#1}}{\partial{#2}}\right\vert_{#3}}
\newcommand{\lagdiff}[3]{\frac{D_{\small #1}{#2}}{D{#3}}}
\newcommand{\ilpdiffcon}[3]{\left.\partial{#1}/\partial{#2}\right\vert_{#3}}
\newcommand{\diffcon}[3]{\left.\frac{d{#1}}{d{#2}}\right\vert_{#3}}
\newcommand{\ildiffcon}[3]{\left. d{#1}/d{#2}\right\vert_{#3}}

%% VECTOR OPERATORS
\newcommand{\bv}[1]{\mbox{{\bf #1}}}
\newcommand{\Div}{\Grad\!\cdot}
\newcommand{\Grad}{\mbox{\boldmath $\nabla$}}
\newcommand{\Curl}{\Grad\!\times}
\newcommand{\delsq}{\nabla^2}

%% MACROS
\newcommand{\I}{\underline{\bv{I}}}
\newcommand{\e}{\textrm{e}}
\newcommand{\khat}{\bv{k}}
\newcommand{\rhf}{\rho^L}
\newcommand{\rhm}{\rho^S}
\newcommand{\rhn}{\rho_0}
\newcommand{\rhmn}{\rho_{m0}}
\newcommand{\rhfn}{\rho_{f0}}
\newcommand{\rbar}{\overline{\rho}}
\newcommand{\drho}{\Delta{\rho}}
\newcommand{\drhon}{\Delta\rho_0}
\newcommand{\vf}{\bv{v}^L}
\newcommand{\vl}{\bv{v}^L}
\newcommand{\vm}{\bv{v}^S}
\newcommand{\vs}{\bv{v}^S}
\newcommand{\vel}{\bv{v}}
\newcommand{\vbar}{\overline{\bv{v}}}
\newcommand{\wbar}{\overline{w}}
\newcommand{\perm}{K}
\newcommand{\trans}{^T}
\newcommand{\DCn}{\Delta C_0}
\newcommand{\cpf}{c_{P,f}}
\newcommand{\cpm}{c_{P,m}}
\newcommand{\cpq}{c_{P,q}}
\newcommand{\cp}{c_{P}}
\newcommand{\omp}{(1-\phi)}
\newcommand{\difn}{\mathcal{D}}
\newcommand{\pcmp}{\mathcal{P}}
\newcommand{\pec}{\textrm{Pe}_C}
\newcommand{\pet}{\textrm{Pe}_T}
\newcommand{\gvec}{\bv{g}}
\newcommand{\gmag}{\vert\gvec\vert}
\newcommand{\ghat}{\hat{\gvec}}
\newcommand{\strr}{\dot{e}}
\newcommand{\ien}{\varepsilon}
\newcommand{\ent}{\mathcal{H}}
\newcommand{\ienbar}{\overline{\ien}}
\newcommand{\entbar}{\overline{\ent}}
\newcommand{\kbar}{\overline{k}}
\newcommand{\aad}{\mathcal{A}}
\newcommand{\stefan}{\mathcal{S}}
\newcommand{\tfo}{T_{Fo}}
\newcommand{\tfe}{T_{Fe}}
\newcommand{\TP}{\mathcal{T}}
\newcommand{\Da}{\textrm{Da}}


\title{Melt ascent and differentiation from the upper mantle to the lower crust} 
\author{Tobias Keller}

\begin{document}
\maketitle \linenumbers \doublespace

\section{Introduction}
In continental arc and rift volcanism erupting lavas are thought to be fed through trans-crustal magma processing systems extending down into the deep crust and mantle lithosphere. The magma source, however, is located even deeper down in the mantle asthenosphere, where decompression by upward mantle flow and/or fluxing by aqueous fluid infiltration lead to partial melting of hot mantle rock. The processes by which mantle melts first ascend into the base of the continental plate above and their rates and spatial distribution remain poorly understood. Current gaps in knowledge are due to, on one hand, the challenges in gaining sufficiently resolved data on these regions by geophysical imaging methods. The materials involved can only be sampled in few locations where lower crustal section have been tilted and eroded to reveal the roots of a transcrustal magmatic system, or through occasional xenoliths, rock pieces brought from depth to the surface in fast ascending magmas. On the other hand, modelling of melt ascent into the base of a plate is challenging due to a range of complex nonlinearities inherent in the problem. Mechanical interactions between melt and host rock are prone to instabilities and localisation feedbacks, leading to narrowly localised zones of enhanced melt flow and rock deformation. Thermochemical reactions between the phases can also lead to localising feedbacks such as dissolution channels solidification chimneys.

One possible way to gain better understanding of melt ascending into the deep crust is to use the chemistry of plutonic and volcanic igneous rocks as records of the dynamic process environments experienced during magma ascent and emplacement. This task is complicated by the fact that geochemical signatures in final rock products are the result of reaction and transport rates acting along a potentially protracted and circuitous trajectory through pressure-temperature-composition space. Interpreting final compositions in terms of mass balancing models, whereby a composition is modelled as a mixture of assumed end-members (e.g., source rock, fractionated crystals, crustal assimilant), is widely practised but results are inherently non-unique. 

In this contribution, we present a coupled thermo-chemical-mechanical model of melt ascent into the base of an overriding plate. The model includes two-phase mechanics including non-Newtonian rheology and decompaction failure, as well as a thermo-chemical model considering the evolution of melt fraction as a function of temperature and major-element composition, as well as tracking compactible and incompatible trace elements, and radiogenic and stable isotopic composition. The geochemical evolution model employs an idealised approach whereby the complex geochemistry involving a dozen major elements, a few minor elements and volatiles, tens of trace elements, and a diversity of radiogenic decay and stable isotope systems are reduced to few idealised components approximating some of the key behaviours of each aspect of geochemistry.


%\begin{figure}[htb]
%  \centering
%  \includegraphics[width=0.6\textwidth]{../figs/Fig1.pdf}
%  \caption{Fig. 1}
%  \label{fig:1}
%\end{figure}


\section{Method}

\subsection{Mechanical model}

Melt ascent is modelled as the porous flow of an incompressible, viscous fluid through a permeable and deformable matrix of incompressible, visco-plastic solid subject to decompaction failure. The non-dimensionalised governing equations expressing conservation of mass and momentum in the two-phase mixture are,
\begin{linenomath*}
\begin{subequations}
\label{eq:governing}
\begin{align}
	\label{eq:moment-mixture}
	\Grad P &= - \Grad p + \Div \underline{\boldsymbol{\tau}} + \phi \mathrm{B} \hat{\mathbf{z}}  \ , \\
	\label{eq:mass-mixture}
	\Div \mathbf{V} &= - \Div \vel \ ,
%	\label{eq:liquid-evo}
%	\dfrac{D_b \phi}{D t} &= \Div (\mathrm{F}-\phi) \mathbf{V}  \ ,
\end{align}
\end{subequations}
\end{linenomath*}
where $\mathbf{V} = [U,V,W]$ is the dynamic velocity, and $P$ the dynamic pressure of the mixture, $\vel = [u,v,w]$ the segregation velocity of the pore fluid, and $p$ the compaction pressure and $\underline{\boldsymbol{\tau}}$ the shear stress tensor of the solid matrix. $\nabla = [\partial/\partial x,\partial/\partial y,\partial/\partial z]$ is the partial spatial derivative. All variables and parameters are functions of position, $\mathbf{x} = [x,y,z]$, and time, $t$, unless indicated otherwise by a subscript nought. The unit vector of the depth coordinate,  $\hat{\mathbf{z}} $, points in the direction of gravity. The dimensionless number $\mathrm{B}$ expresses the magnitude of buoyancy forces to tensile strength of the aggregate (more details below). 

%and $D_b/Dt = \partial/\partial t + \mathbf{V}_b \cdot \Grad$ the Lagrangian time derivative in the reference frame of the background flow field (see next paragraph)

The dynamic velocity and pressure are each reduced by an invariant background field with constant spatial gradients. The dynamic pressure, $P$, is the Stokes pressure of the mixture reduced by a lithostatic background field, $P_b$, with $\Grad P_\mathrm{lith} = [0,\rho_0 g_0]$, where $\rho_0$ is the constant solid matrix density, and $g_0$ the gravity constant. The dynamic velocity is the Stokes velocity of the mixture reduced by a background flow field, $\mathbf{V}_b$, which can assume both pure- and simple-shear flow patterns adjusted by dimensionless factors $\mathrm{Pu}$, and $\mathrm{Si}$ which control the magnitude of shear stress induced by backround deformation relative to the tensile strength of the aggregate.

The non-dimensionalised closures for segregation velocity, $\vel$, compaction pressure, $p$, shear-stress tensor, $\underline{\boldsymbol{\tau}}$, and visco-plastic matrix rheology are,
\begin{linenomath*}
\begin{subequations}
\label{eq:nondim-coeffs}
\begin{align}
	\label{eq:nondim-coeffs-segr}
	\vel &= -K \left( \Grad P + \mathrm{B} \hat{\mathbf{z}} \right) \ , \\
	\label{eq:nondim-coeffs-cpct}
	p &= -\dfrac{\eta}{\phi} \Div \mathbf{V} \ , \\
	\label{eq:nondim-coeffs-stress}
	\underline{\boldsymbol{\tau}} &= \eta \, \underline{\dot{\boldsymbol{\varepsilon}}} \ , \\
	\label{eq:nondim-darcy}
	K &= \left(\dfrac{\phi}{\phi_0}\right)^m \exp \left(-E^\ell \left[ \dfrac{1}{T} - \dfrac{1}{T_0} \right] \right) \ , \\
	\label{eq:nondim-coeffs-viscoplast}
	\eta &= \min\left( \exp \left( E^s \left[ \dfrac{1}{T} - \dfrac{1}{T_0} \right] -\dfrac{\lambda} {\mathrm{F}} (\phi-\phi_0) \right) \left(\dfrac{\dot{\varepsilon}}{\dot{\varepsilon}_b} \right)^{-n} \: , \:\: \dfrac{\tau_y}{\dot{\varepsilon}} + \eta_\mathrm{min} \right) \ , \\
	\label{eq:nondim-coeffs-yield}
	\tau_y &= 1 + p \ .
\end{align}
\end{subequations}
\end{linenomath*}
$\hat{\mathbf{z}}$ is the unit vector pointing along the depth coordinate. The shear strain rate tensor, $\underline{\dot{\boldsymbol{\varepsilon}}}$, its components, and magnitude are defined as,
\begin{linenomath*}
\begin{subequations}
\label{eq:nondim-strainr}
\begin{align}
	\label{eq:nondim-strainr-comps}
	\underline{\dot{\boldsymbol{\varepsilon}}} &= \frac{1}{2} \left(\Grad \mathbf{V} + \Grad \mathbf{V}^T\right) -\frac{1}{3} \Div \mathbf{V} \underline{\mathbf{I}} \ , \\
	\label{eq:nondim-strainr-magn}
	\dot{\varepsilon} &= \sqrt{\dfrac{1}{2} \underline{\dot{\boldsymbol{\varepsilon}}} :  \underline{\dot{\boldsymbol{\varepsilon}}} } \ . 
\end{align}
\end{subequations}
\end{linenomath*}
The rheological coefficients are centred on the reference melt fraction, $\phi_0$, temperature, $T_0$, and strain-rate $\dot{\varepsilon}_b$. The matrix creep viscosity is weakened exponentially by the presence of melt with a factor $\lambda=30$, by inverse temperature with the solid activation energy $E^s$,  and is softening as a powerlaw of the strain rate magnitude, $\dot{\varepsilon}$, with a non-Newtonian powerlaw exponent $n=2/3$. The reference strain-rate for the non-Newtonian powerlaw is determined by the applied pure- and/or simple-shear background deformation, $\dot{\varepsilon}_b = |\mathrm{Pu}| + |\mathrm{Si}|$. Hence, the dimensionless shear viscosity, $\eta$, at $\phi = \phi_0$, $T = T_0$, and $\dot{\varepsilon} = \dot{\varepsilon}_b$ is equal to unity, and the compaction viscosity equal to $\phi^{-1}$. Decompaction failure is implemented using an effective-viscosity approach limiting shear stress as a function of compaction pressure so as not to exceed the tensile failure criterion, $\tau_y$ (see Fig. \ref{fig:yield}). A minimum yield viscosity, $\eta_\mathrm{min}$, is added to regularise the problem by imposing a minimum length scale of localised failure zones.

To highlight the role of decompaction failure, we have scaled pressures and stresses by the tensile strength of the matrix, $p_o = \sigma_{T,0}$ to non-dimensionalise the problem. The dimensionless number $\mathrm{B} = \Delta p_0 / \sigma_{T,0}$ expresses the relative magnitude of the buoyancy-related phase pressure difference, $ \Delta p_0 = \Delta \rho_0 g_0 \delta_0$, relative to the tensile strength scale. The former is a natural scale of the Stokes-Darcy problem proportional to the matrix-melt density difference, $\Delta \rho_0$. Velocities are scaled by the Darcy speed scale $w_0 = K_0 p_0/\ell_0$, and length by $\ell_0 = \sqrt{\eta_0 K_0}$, a natural scale of the problem related to the segregation-compaction length, $\delta_0 =  \sqrt{\phi_0^{-1}} \ell_0$, where $\eta_0$ is the characteristic matrix shear viscosity, and $K_0 = a_0^2 \phi_0^m / (b_0 \mu_0)$ the Darcy coefficient scale, with $a_0$ the characteristic matrix grain size, $b_0 = 100$ a geometrical constant, and $\mu_0$ the characteristic melt viscosity.

\subsection{Thermo-chemical model}

The thermo-chemical evolution by coupled reaction-transport processes in the partially molten mixture is modelled by a set of equations conserving energy, phase, and component mass, and evolving trace and isotopic composition,
\begin{linenomath*}
\begin{subequations}
\label{eq:nondim-coeffs}
\begin{align}
	\label{eq:nondim-phase}
	\dfrac{\partial \phi}{\partial t}  &= \Div (1-\phi) \mathbf{v}^s +  \Gamma \ , \\
	\label{eq:nondim-energy}
	\dfrac{\partial T}{\partial t} &= - \bar{\mathbf{v}} \cdot \Grad T + \dfrac{ \boldsymbol{\nabla}^2 T }{\mathrm{Pe}_T} - \dfrac{\Gamma}{\mathrm{St}} \ , \\
	\label{eq:nondim-major}
	\dfrac{\partial \bar{c}_\mathrm{maj}}{\partial t} &= - \Div \overline{c_\mathrm{maj} \mathbf{v}} + \dfrac{ \boldsymbol{\nabla}^2 \bar{c}_\mathrm{maj} }{\mathrm{Pe}_c}  \ , \\
	\label{eq:nondim-trace}
	\dfrac{\partial \bar{c}_\mathrm{tra}}{\partial t} &= - \Div  \overline{c_\mathrm{tra} \mathbf{v}} + \dfrac{ \boldsymbol{\nabla}^2 \bar{c}_\mathrm{tra} }{\mathrm{Pe}_c}  \ , \\
	\label{eq:nondim-radiogen-parent}
	\dfrac{\partial \bar{c}_\mathrm{irp}}{\partial t} &= - \Div  \overline{c_\mathrm{irp} \mathbf{v}} + \dfrac{ \boldsymbol{\nabla}^2 \bar{c}_\mathrm{irp} }{\mathrm{Pe}_c}  - \Gamma_D  \ , \\
	\label{eq:nondim-radiogen-daughter}
	\dfrac{\partial \bar{c}_\mathrm{ird}}{\partial t} &= - \Div  \overline{c_\mathrm{ird} \mathbf{v}} + \dfrac{ \boldsymbol{\nabla}^2 \bar{c}_\mathrm{ird} }{\mathrm{Pe}_c}  + \Gamma_D  \ , \\
	\label{eq:nondim-stable-matrix}
	\dfrac{\partial c_\mathrm{ist}^s}{\partial t} &= - \mathbf{v}^s \cdot \Grad c_\mathrm{ist}^s + \dfrac{ \boldsymbol{\nabla}^2 c_\mathrm{ist}^s }{\mathrm{Pe}_c}  - (c_\mathrm{ist}^\Gamma - c_\mathrm{ist}^s) \dfrac{\Gamma}{1-\phi}  \ , \\
	\label{eq:nondim-stable-melt}
	\dfrac{\partial c_\mathrm{ist}^\ell}{\partial t} &= - \mathbf{v}^\ell \cdot \Grad c_\mathrm{ist}^\ell + \dfrac{ \boldsymbol{\nabla}^2 c_\mathrm{ist}^\ell}{\mathrm{Pe}_c} + (c_\mathrm{ist}^\Gamma - c_\mathrm{ist}^\ell) \dfrac{\Gamma}{\phi}  \ .
\end{align}
\end{subequations}
\end{linenomath*}

The melt fraction, $\phi$, evolves as a consequence of compaction, that is, divergence in the solid velocity field, $\mathbf{v}_s$, and (volumetric) melting rate, $\Gamma$. Sensible heat characterised by the temperature relative to the interval from lowest to highest melting point evolves as a function of advection by the mixture velocity field, $\bar{\mathbf{v}} = \phi \mathbf{v}^\ell + (1-\phi) \mathbf{v}^s$, diffusion governed by the inverse of the thermal Peclet number, $\mathrm{Pe}_T$, and sensible to latent heat conversion upon melting governed by the Stefan number, $\mathrm{St}$. The major element composition of the mixture, $\bar{c_\mathrm{maj}} = \phi c_\mathrm{maj}^\ell + (1-\phi) c_\mathrm{maj}^s$ evolves by divergence in the chemical flux, $c_\mathrm{maj}^s \mathbf{v}^s  + \phi c_\mathrm{maj}^\ell \mathbf{v}^\ell$, and chemical diffusion governed by the chemical Peclet number, $\mathrm{Pe}_c$. Note that chemical diffusivity of major elements is very slow on crustal length scale, hence chemical diffusion could be safely neglected in these models. We include it here for the purpose of regularisation to avoid growth of overly sharp compositional boundaries that can become numerically problematic for advection schemes. The phase compositions, $c_\mathrm{maj}^{\ell,s}$ are functions of temperature and mixture composition defined by a phase diagram. Total mass conservation is observed by only solving for one phase, assuming that the mixture is saturated and both phase fractions sum to unity, and only solving for one major element component assuming that the two components also sum to unity (i.e., the lever rule applies).

The geochemical evolution of trace and isotopic compositions does not need to observe total mass conservation since the elements involved are assumed to be present in trace quantities only. Hence, equations for all components are solved and no unity sum constraints apply. The mixture trace composition, $\bar{c_\mathrm{tra}}$ follows the same form of conservation law as major element composition, with phase trace element compositions, $c_\mathrm{tra}^{\ell,s}$, are set by partition coefficients, $K_\mathrm{tra} = c_\mathrm{tra}^{s}/c_\mathrm{tra}^{\ell}$, here held constant for simplicity. Mixture concentrations of a set of parent and daughter isotopes linked by a radioactive decay system follow the same form of conservation law as major and trace components but are linked by the decay rate, $\Gamma_D = \bar{c}_\mathrm{irp} / (\ln(2) D_\mathrm{1/2})$, proportional to the concentration of the parent isotope, $\bar{c}_\mathrm{irp}$, and the decay number, $D_\mathrm{1/2}$, which is the dimensionless half-life. Stable isotopic composition of the melt and matrix phases are assumed to evolve only by advection on the phase velocity field and mass transfer between phases by phase-change reaction. Diffusion terms are added for the purpose of regularisation.

\subsection{Numerical Implementation}

The model is implemented in two dimension on a square domain of $L \times L$ non-dimensional size. The initial condition on temperature is a halfspace  cooling solution representing a cool plate over a near-isothermal mantle. The initial major element concentration is a smoothed random noise to introduce some heterogeneity. The initial melt fraction is set to the equilibrium value given the temperature and bulk composition as specified by an idealised peritectic-eutectic phase diagram. The boundary conditions are free slip and insulating on the sides and free in/outflow and isothermal on the top and base. The side boundaries can be switched to periodic in models where horizontal simple shear dominates the background velocity.

The numerical implementation discretises the governing equations on a regular square grid of 500$\times$500 finite-volume cells using a centred staggered-grid finite-difference scheme. Advection terms are implemented using flux-conservative centred or upwind-biased Fromm methods. The equations are solved using a damped Richardson iterative scheme. For reasons of numerical stability the shear visco-plasticity, $\eta$, and the phase reaction rate, $\Gamma$, are regularised by applying a small amount of Laplacian smoothing and lagged iteratively to avoid numerical oscillations. The mechanical part of the numerical algorithm was benchmarked against a manufactured analytical solution and shows second order error convergence with increasing spatial resolution. The simulation code was developed and tested in Matlab version R2020a and is openly available in on github.com/kellertobs/ivrea.


%\bibliographystyle{abbrvnat}
%\bibliography{notes}

%\appendix
%\section{Governing equations}
%\begin{linenomath*}
%\begin{subequations}
%\label{eq:governing}
%\begin{align}
%	\label{eq:moment-mixture}
%	\Grad P &= -\Grad P_\Delta + \Div \underline{\boldsymbol{\tau}}  \ , \\
%	\label{eq:mass-mixture}
%	\Div \vel &= - \Div \vel_\Delta \ , \\
%	\label{eq:liquid-evo}
%	\dfrac{\partial \phi}{\partial t} &= \Div (1-\phi) \vel \ . 
%\end{align}
%\end{subequations}
%\end{linenomath*}
%The variables are the solid velocity, $\vel = \vel^s$, acting as reference for the segregation velocity of the liquid, $\vel_\Delta = \phi (\vel^\ell - \vel^s)$, the dynamic (non-lithostatic) liquid pressure, $P = P^\ell - \rho^s g$, acting as reference for the compaction pressure, $P_\Delta = (1-\phi)(P^s-P^\ell)$, where $\phi$ is the volume fraction of the liquid phase in the saturated phase mixture, and superscripts $s$ and $\ell$ denote properties of the solid and liquid phases, respectively. $\underline{\boldsymbol{\tau}}$ is the shear stress tensor of the mixture. Buoyancy forces in the momentum equation \eqref{eq:moment-mixture} are of order $\phi \ll 1$ and are therefore assumed to be negligible here.
%
%\subsection{Constitutive relations}
%The governing equations \eqref{eq:governing} require constitutive relations for the segregation velocity, compaction pressure, and shear stress,
%\begin{linenomath*}
%\begin{subequations}
%\label{eq:constitutive}
%\begin{align}
%	\label{eq:constitutive-segr}
%	\vel_\Delta &= - K \left( \Grad P + \Delta \rho \gvec \right) \ , \\
%	\label{eq:constitutive-comp}
%	P_\Delta &= - \zeta \, \Div \vel \ , \\
%	\label{eq:constitutive-stress}
%	\underline{\boldsymbol{\tau}} &= \eta \, \underline{\mathbf{D}} \ .
%\end{align}
%\end{subequations}
%\end{linenomath*}
%The material parameters are the phase density difference, $\Delta \rho = \rho^s - \rho^\ell$, the segregation coefficient, $K$, and the effective shear and compaction viscosities, $\eta$ and $\zeta$. $\underline{\mathbf{D}} = \dfrac{1}{2} (\Grad \vel + [\Grad \vel]^T) - \dfrac{1}{3} \Div \vel \I$ is the deviatoric symmetrical velocity gradient or shear strain-rate tensor.
%
%The Darcy segregation coefficient is,
%\begin{linenomath*}
%\begin{align}
%	\label{eq:coeff-segr}
%	K =  \dfrac{a_0 \phi^n}{b \mu_0} , 
%\end{align}
%\end{linenomath*}
%with $a_0$ the characteristic grain size of the matrix [m], $b$ a dimensionless geometric factor of order $\sim 100$, $n$ the permeability powerlaw coefficient, and $\mu_0$ the liquid viscosity, which we hold constant here.
%
%\subsection{Non-dimensional governing equations}
%To non-dimensionalise the governing equations \eqref{eq:governing} including their constitutive relations \eqref{eq:constitutive}, we introduce characterstic scales for pressure and stress $[P,P_\Delta,\underline{\boldsymbol{\tau}},\tau_\mathrm{y}] = p_0$, velocity, $[\vel,\vel_\Delta] = w_0$, liquid fraction $[\phi] = \phi_0$, length $\delta_0$ to scale spatial derivatives, $[\Grad] = 1/\delta_0$, time $[t] = \delta_0/w_0$, viscosities $[\eta] = \eta_0$, and $[\zeta] = \eta_0/\phi_0$, Darcy coefficient, $[K] = k_0 \phi_0^m / \mu_0$, density difference $[\Delta \rho] = \Delta \rho_0 = \rho^s_0 - \rho^\ell_0$, and gravity, $[\mathbf{g}] = g_0$.
%
%Upon inspection the following emerge as the characteristic speed, pressure, and length scales of the problem, 
%\begin{linenomath*}
%\begin{subequations}
%\label{eq:scales}
%\begin{align}
%	\label{eq:scales-pressure}
%	p_0 &= \Delta \rho_0 g_0 \delta_0 \ , \\
%	\label{eq:scales-speed}
%	w_0 &= \dfrac{K_0 p_0}{\delta_0} \ , \\
%	\label{eq:scales-length}
%	\delta_0 &= \sqrt{\eta_0 K_0} \ .
%\end{align}
%\end{subequations}
%\end{linenomath*}
%
%\subsection{Visco-elastic rheology with decompaction failure}
%The deformation of a granular solid at the microscopic, granular scale is a superposition of the internal deformation of individual grains and the cohesive-frictional interactions between adjacent grains. We model the effective macroscopic deformation as viscous creep subject to tensile plastic failure. The former is dominated by fluid-like deformation at strain rates proportional to the stress applied. We assume that the contribution of solid-like elastic strain is negligible in the relatively ductile porous media we are interested in. Plastic failure occurs when the stress state locally reaches the yield stress. At this point, the material will continue deforming at increased strain rates while stress remains limited; hence, the effective deformational strength of the material is reduced, and deformation tends to become strongly localised.
%
%Here we employ a simplified representation of the Griffith-Murrell failure criterion as a tensile cutoff. The criterion only admits stress states whos Moh-Coulomb circle with radius of shear stress magnitude, $|\boldsymbol{\tau}|$, and origin of effective mean normal stress, $\langle \sigma_\mathrm{eff} \rangle$, is bounded to the left by the tensile strength, $-\sigma_T$. The shear stress magnitude is related to the second norm of the stress tensor, $|\boldsymbol{\tau}|$
%It gives the maximum permitted shear stress magnitude as a function of the effective confining stress, which is the mean or isotropic stress magnitude reduced by the pore fluid pressure. Hence, if the pore fluid has a pressure similar to the mean stress in the granular matrix, the effective pressure, $P_\Delta$ in our notation, will be near zero. It will be negative where the liquid becomes overpressured.
%
%We write the tensile failure criterion as,
%\begin{linenomath*}
%\begin{align}
%  \label{eq:failure-criterion}
%  |\underline{\boldsymbol{\tau}}| \leq \tau_\mathrm{y} = \sigma_T + P_\Delta \ .
%\end{align}
%\end{linenomath*}
%where the yield strength, $\tau_\mathrm{y}$, sets the maximum shear stress magnitude, $|\underline{\boldsymbol{\tau}}|$, with tensor magnitude defined as $ | \underline{(\cdot)} | = \sqrt{\frac{1}{2} \underline{(\cdot)} : \underline{(\cdot)}}$, as a function of the tensile rock strength, $\sigma_T$, and the effective pressure, $P_\Delta$.
%
%We impose the failure criterion by choosing an effective yield viscosity that enforces the stress limit,
%\begin{linenomath*}
%\begin{align}
%	\label{eq:strainr-magnitude}
%	\eta_\mathrm{y} = \min\left( \eta_\mathrm{v} \: , \:\: \dfrac{\tau_\mathrm{y}}{|\underline{\mathbf{D}}|} \right) \ ,
%\end{align}
%\end{linenomath*}
%where the shear viscosity of the porous medium follows a  liquid- and strain-rate weakened closure,
%\begin{linenomath*}
%\begin{align}
%	\label{eq:visc-shear}
%	\eta_\mathrm{v} &= \eta_0 \exp(\lambda (\phi-\phi_0)) \left(|\underline{\mathbf{D}}| / \varepsilon_0 \right)^{-n}  \ .
%\end{align}
%\end{linenomath*}
%$\eta_0$ is the reference viscosity set at the reference liquid fraction, $\phi_0$, and the reference strain rate magnitude, $\varepsilon_0$, $27 \leq \lambda \leq 30$ an exponential liquid-weakening factor, $0 \leq n \leq 0.75$ the strain-rate weakening powerlaw exponent, and $|\underline{\mathbf{D}}|$ the magnitude of the shear strain-rate tensor.
%
%We follow a similar procedure to choose constitutive relation for compaction pressure above \eqref{eq:constitutive}, with the compaction visco-plasticity a taken as the shear visco-plasticity divided by the porosity,
%\begin{linenomath*}
%\begin{subequations}
%\begin{align}
%	 \label{eq:visc-elast-cmpeffvisc}
%	 \zeta &= \dfrac{\eta}{\phi}  \ .
%\end{align}
%\end{subequations}
%\end{linenomath*}
%Hence, compaction deformation is equally subject to melt- and strain-rate weakening, as well as decompaction failure.

\end{document}
